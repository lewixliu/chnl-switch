\begin{abstract}

Celem niniejszej pracy jest implementacja narzędzia pomiarowego \emph{hop-sniffer} służącego do obserwacji zjawisk zachodzących podczas komunikacji w standardzie \emph{802.11}. Nacisk kładziony jest na opóźnienie wprowadzane przez procedurę roamingu stacji klienckiej. Powstały program użyty jest do przeprowadzenia scenariuszy pomiarowych, których wyniki służą do identyfikacji czynników wpływających na zwiększenie opóźnienia. Ze względu na charakter mierzonych wartości, analizie podlega ich użyteczność w związku z systemami czasu rzeczywistego. \\ \vspace{12pt}

Słowa kluczowe: standard 802.11, roaming, opóźnienie ramek, systemy czasu rzeczywistego \hfill
\end{abstract}

\vspace{24pt}

\renewcommand{\abstractname}{MEASUREMENT OF \emph{802.11} FRAMES LATENCY- \emph{HOP-SNIFFER} TOOL}

\begin{abstract}

The main target of this thesis is the implementation of measurement tool \emph{hop-sniffer} used for observation of 802.11 standard communication events. Emphasis is on the overhead of client station roaming procedure. The resulting program is used to carry out the measurement scenarios. The results are used to identify factors influencing the increase in delays. Due to the nature of the measured values their usefulness in conjunction with real-time systems is analyzed.\\ \vspace{12pt}

Keywords: 802.11 standard, roaming, frame latency, real-time systems \hfill
\end{abstract}

