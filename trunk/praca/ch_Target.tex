% Cel pracy. Open source, systemy cz. rzeczywistego, 802.11.

Celem niniejszej pracy inżynierskiej jest specyfikacja wymagań i implementacja narzędzia pomiarowego \emph{hop-sniffer} umożliwiającego obserwacje wybranych zjawisk zachodzących podczas komunikacji systemów w standardzie 802.11. Główny nacisk kładziony jest na pomiar zależności czasowych między zdarzeniami charakteryzującymi dany scenariusz komunikacyjny. Poprzez zdarzenie rozumiem fakt nadania lub odebrania ramki 802.11. Ramki są podstawowym elementem protokołu komunikacyjnego, więc możliwość obrazowania zależności czasowych między nimi daje szansę  ustalenia trwania dowolnych zjawisk charakteryzujących 802.11.

Skupienie na pomiarze parametrów czasowych wynika z przyświecającego pracy celu analizy zastosowania komunikacji bezprzewodowej w wymianie danych systemów czasu rzeczywistego. Systemy takie w swoim obrębie pracują w sposób deterministyczny i zgodny z przyjętymi założeniami. Problem pojawia się w sytuacji, gdy nadchodzi potrzeba ustanowienia łącza komunikacyjnego między nimi. W celu zachowania założeń opóźnienia wprowadzane przez takie łącze muszą spełniać te same rygorystyczne wymagania co same systemy. Możliwość pomiarów oferowana przez program \emph{hop-sniffer} jest pomocna nie tyle w odpowiedzi na pytanie, czy komunikacja w standardzie 802.11 jest deterministyczna, lecz jakie mogą być przybliżone czasy trwania wybranych zjawisk w ustalonych warunkach medium. 

Z pośród wybranych scenariuszy w komunikacji bezprzewodowej praca ta skupia się głównie na roamingu 802.11 stacji klienckiej i będącym jego integralną częścią zjawisku przełączania kanału radiowego. Zjawisko to jest ważne głownie z perspektywy stacji mobilnych, których interfejsy radiowe, w celu zachowania łączności, są zmuszone do zmian częstotliwości pracy zgodnie z kanałem działania punktu dostępowego obsługującego aktualnie odwiedzany obszar. Sformułowanie sposobu pomiaru czasu trwania roamingu stacji klienckiej posłuży mi do wyznaczenia wymagań stawianych aplikacji \emph{hop-sniffer}. Wymagania te są podstawą do wyboru technik programistycznych, używanych bibliotek i środowiska działania programu.

Ostatecznie niniejsza praca wymaga wykorzystania aplikacji \emph{hop-sniffer} w realizacji uprzednio sformułowanej procedury pomiarowej w celu wyciągnięcia wniosków co do czasu przełączania kanału radiowego podczas roamingu stacji klienckiej w standardzie 802.11. Wnioski dotyczą głównie wpływu zastosowanych ustawień i wybranego środowiska pod kątem możliwego zastosowania w systemach czasu rzeczywistego.
