\documentclass[11pt,a4paper,oneside]{report}
\usepackage[utf8]{inputenc}
\usepackage{polski}
\usepackage[polish]{babel}
\usepackage{indentfirst}
\usepackage{url}

\usepackage{amsmath}
\usepackage{color}
\usepackage{listings}
\usepackage{graphicx}
\usepackage{hyperref}

%\newcommand{\bib}[1]{\cite{#1}

%\lstlistoflistings
\definecolor{darkgray}{rgb}{0.95,0.95,0.95}
\definecolor{dkgreen}{rgb}{0,0.6,0}
\definecolor{mauve}{rgb}{0.58,0,0.82}

\lstset{language=C}
\lstset{basicstyle=\footnotesize}
\lstset{backgroundcolor=\color{darkgray}}
\lstset{numbers=left, numberstyle=\tiny, stepnumber=2, numbersep=5pt}
\lstset{keywordstyle=\color{red}\bfseries\emph}
\lstset{commentstyle=\color{dkgreen}}
\lstset{stringstyle=\color{mauve}}

\begin{document}
\title{ \emph{hop-sniffer}: Narzędzie pomiaru opóźnień w standardzie \emph{802.11}. }
\author{ Marcin Harasimczuk }
\maketitle

\tableofcontents

% Empty page.
\newpage
\thispagestyle{empty}
\mbox{}
\bibliographystyle{plain}

\section{Cel pracy.}
% Cel pracy. Open source, systemy cz. rzeczywistego, 802.11.

Rozwój technologii bezprzewodowych powoduje ciągły wzrost zainteresowania standardem 802.11. Rozwiązania, które wykluczają konieczność użycia drogiego i często niewygodnego okablowania są oczywistym wyborem dla zastosowań przemysłowych, zwłaszcza jeśli w grę wchodzi użycie mobilnych stacji lub agentów. Ze względu na charakterystykę informatycznych systemów przemysłowych, które działają na styku z fizycznymi zjawiskami, często zachodzi konieczność stosowania w nich systemów czasu rzeczywistego. Rozwój systemów operacyjnych realizujących swoje zadania w ograniczonym i deterministycznym czasie stał się również celem środowisk open-source, czego dowodem jest istnienie takich projektów jak RTAI, Xenomai, czy ciągłe udoskonalanie jądra systemu Linux pod kątem redukcji niedeterministycznych opóźnień. 

Analizując publikacje naukowe z ostatnich lat można dostrzec dużą liczbę prac poruszających kwestię wymiany danych między systemami czasu rzeczywistego w medium bezprzewodowym (przykładowo \cite{pub:AdvAp}, \cite{pub:SpaceTimeCoding}). Dobrym przykładem systemu z agentami mobilnymi jest projekt \emph{ECO-Mobilność} \cite{www:ECO-Mob} wykorzystujący pojazdy \emph{PRT} (ang. \emph{Personal Rapid Transport}). Łączenie tych dwóch technologii wymusza skupienie większej uwagi na parametrach czasowych jakimi charakteryzuje się komunikacja w standardzie 802.11. Uważam, że potrzebne jest rozwiązanie pozwalające na przeprowadzenie pomiarów opóźnień wymiany danych dla scenariuszy posiadających ograniczenia czasowe wynikające z potrzeby utrzymania połączenia \cite{pub:VirtualAP}. 

Celem niniejszej pracy inżynierskiej jest wytworzenie narzędzia pomiarowego \emph{hop-sniffer} umożliwiającego obserwacje wybranych zjawisk zachodzących podczas komunikacji systemów w standardzie 802.11. Główny nacisk kładziony jest na pomiar zależności czasowych między zdarzeniami charakteryzującymi dany scenariusz komunikacyjny. Poprzez zdarzenie rozumiem fakt nadania lub odebrania ramki 802.11. Ramki są podstawowym elementem protokołu komunikacyjnego, więc możliwość obrazowania zależności czasowych między nimi daje szansę ustalenia trwania dowolnych zjawisk charakteryzujących 802.11.

Z pośród wybranych scenariuszy w komunikacji bezprzewodowej praca ta skupia się głównie na roamingu 802.11 stacji klienckiej i będącym jego integralną częścią zjawisku przełączania kanału radiowego. Zjawisko to jest ważne głownie z perspektywy stacji mobilnych, których interfejsy radiowe, w celu zachowania łączności, są zmuszone do zmian częstotliwości pracy zgodnie z kanałem działania punktu dostępowego obsługującego aktualnie odwiedzany obszar. Z punktu widzenia komunikacji systemów czasu rzeczywistego ważne jest określenie ograniczeń czasowych ze względu na fakt zerwania połączenia agenta z systemem podczas przełączania między punktami dostępowymi. Sformułowanie sposobu pomiaru czasu trwania roamingu stacji klienckiej posłuży mi do wyznaczenia wymagań stawianych aplikacji \emph{hop-sniffer}. Wymagania te są podstawą do wyboru technik programistycznych, używanych bibliotek i środowiska działania programu.



\section{Wprowadzenie do dziedziny.}
% Wstęp ogólny łączący 802.11 i systemy czasu rzeczywistego.

\section{Problemy systemów czasu rzeczywistego.}

Badanie komunikacji systemów czasu rzeczywistego wymaga analizy dostępnych rozwiązań na poziomie oprogramowania. Jest to czynność niezbędna ze względu na różnice w implementacji i stosowane techniki programistyczne. W swojej pracy skupiam się na rozwiązaniach \emph{open-source} i systemie \emph{Linux} ze względu na dostępność, zgodność ze standardem POSIX oraz otwarty kod źródłowy.

Systemy z rodziny \emph{open-source} charakteryzują się możliwością stopniowania wsparcia dla procesów czasu rzeczywistego (\cite{wiki:RTLinux}). Zaczynając od podstawowej dystrybucji systemu \emph{Linux}, poprzez różnorodne opcje konfiguracyjne jądra w wersji 2.6 i kończąc na koncepcji współdzielenia zasobów sprzętowych. 

Niniejszy rozdział poświęcam na wprowadzenie do problemów systemów czasu rzeczywistego. Ze względu na tematykę pracy koncentruję się również na kwestii komunikacji sieciowej (implementacji stosu IP). 

\subsection{System operacyjny Linux ( wersja jądra 2.6 ).}

Za analizą zastosowania systemu Linux jako systemu czasu rzeczywistego przemawia jego szeroka dostępność i niski koszt zastosowania. Aplikacje pracujące w reżimie czasu rzeczywistego mogą być pisane zgodnie ze standardem POSIX co wyklucza dodatkowy narzut związany z przyswajaniem nowych interfejsów programistycznych. 

Jądro w wersji 2.4, ze względu na zastosowanie BKL (ang. Big Kernel Lock) wymuszało sekwencyjne wykonanie procesów działających w jego kontekście. BKL jest globalnym \emph{spin-lock'iem} zajmowanym przez proces, który zaczyna wykonywać kod jądra (np. w wyniku wywołania systemowego) i zwalnianym po powrocie do przestrzeni użytkownika. Takie podejście zapewnia, że w kontekście jądra może wykonywać się tylko jeden wątek. Sekwencyjność całkowicie wyklucza możliwość zastosowania jako system czasu rzeczywistego. 

Wersja 2.6 znacząco poprawia ten stan rzeczy. Dzięki lokalnemu blokowaniu zasobów wątek jądra może zostać wywłaszczony tylko w ściśle określonych miejscach. Zmniejszanie opóźnień odbywa się poprzez systematyczne zastępowanie \emph{spin-lock'ów} blokadami typu \emph{mutex}. \emph{Mutex} pozwala na lepsze wykorzystanie czasu procesora, gdyż wątek oczekujący na wejście do sekcji krytycznej nie wykonuje aktywnego oczekiwania. Mechanizm \emph{spin-lock} jest lepszym wyborem w sytuacji kiedy jest pewne, że narzut związany z przełączaniem kontekstu jest większy niż przewidywany czas aktywnego oczekiwania. \emph{Spin-lock} jest zatem dobrym rozwiązaniem jeśli mamy do czynienia z ograniczoną współbieżnością, w przeciwnym przypadku powoduje duże straty czasu procesora w skutek aktywnego oczekiwania.

Podobnie jak w wersji 2.4 istnieje ryzyko długich opóźnień, jeśli zadanie o niskim priorytecie zablokuje obsługę przerwań.

Jądro w wersji 2.6 wprowadza nowy algorytm szeregowania procesów nazwany po prostu \emph{O(1)}. Algorytm ten zapewnia, że czas szeregowania nie zależy od ilości procesów. Nie zmienia to faktu, że proces nie będący procesem czasu rzeczywistego może z powodzeniem zablokować możliwość wywłaszczenia blokując obsługę przerwań.

Poważniejsze zmiany w jądrze systemu Linux dostępne są po wybraniu odpowiednich opcji kompilacji. Każda kolejna opcja zwiększa granulację blokad w kodzie jądra co przekłada się na zwiększenie liczby punktów, w których może ono zostać wywłaszczone. Zmiany te, w połączeniu z jawnym oznaczeniem przez programistę zadań pracujących w reżimie czasu rzeczywistego, mają wpływ na parametry opisujące działanie planisty. Pogorszeniu może ulec przepustowość (ang. throughput), rozumiana jako ilość procesów, które kończą swoją pracę na jednostkę czasu. Parametr ten ulega pogorszeniu, gdyż np. polityka SCHED\_FIFO nie obsługuje podziału czasu (ang. time-slicing). Z drugiej strony można oczekiwać poprawy wydajność (ang. scheduler efficiency ) rozumianej jako parametr odwrotnie proporcjonalny do opóźnień wprowadzanych przez planistę. Wydajność może wzrosnąć, gdyż SCHED\_FIFO pozwala na ustalenie stałych priorytetów co w niektórych przypadkach znacząco przyspiesza harmonogramowanie. Obecnie dostępne są następujące opcje konfiguracyjne, przy czym ostatnia z nich wymaga zastosowania łatki: 

\begin{itemize}
\item CONFIG\_PREEMPT\_VOLUNTARY
\item CONFIG\_PREEMPT
\item CONFIG\_PREEMPT\_RT
\end{itemize}

Opcja PREEMPT\_RT wzbogaca jądro o następujące możliwości:

\begin{itemize}
\item Możliwość wywłaszczenia w sekcjach krytycznych
\item Możliwość wywłaszczenia kodu obsługi przerwań
\item Wywłaszczalne obszary \emph{blokowania przerwań}
\item Dziedziczenie priorytetów dla semaforów i \emph{spin-locków} wewnątrz jądra
\item Opóźnione operacje
\item Techniki redukcji opóźnień
\end{itemize}

Po zastosowaniu łatki większość kodu obsługi przerwań wykonuje się w kontekście procesu. Wyjątkiem są przerwania związane z zegarem CPU (np. \emph{sheduler\_tick()}). Zastosowane zmiany powodują, że zadanie wykonujące \emph{spin\_lock()} może zrzec się czasu procesora, a co za tym idzie nie powinno działać przy zablokowanych przerwaniach (pojawia się zagrożenie blokadą). Jako rozwiązanie problemu przyjęto opóźnianie operacji, które nie mogą być wykonane przy zablokowanych przerwaniach do czasu ich odblokowania. Dodatkowe techniki redukcji opóźnień polegają przykładowo na rezygnacji z używania niektórych instrukcji MMX związanych z architekturą x86 (wyselekcjonowano instrukcje uznane za zbyt długie).

\subsection{Techniki programistyczne w standardzie POSIX.}

W środowisku Linux tworzenie aplikacji czasu rzeczywistego polega na przemyślanym przeciwdziałaniu najczęstszym przyczynom długich opóźnień. Jako podstawowe przyczyny opóźnień wyróżniam:

\begin{itemize}
\item Brak stron kodu, danych i stosu związanych z aplikacją w pamięci (ang. \emph{page fault})
\item Opóźnienia powstałe w skutek optymalizacji wprowadzanej przez kompilator (np. \emph{copy-on-write})
\item Dodatkowy czas potrzebny na tworzenie nowych wątków, z których korzysta aplikacja
\end{itemize}

Podstawowym krokiem ku zapewnieniu, że kod aplikacji uzyska czas procesora tak szybko jak to potrzebne jest jawne oznaczenie danego procesu. Oznaczenie odbywa się poprzez wybranie trybu kolejkowania i priorytetu dla zadania. W celu oznaczenia wykorzystuję funkcję \emph{sched\_setscheduler()}, która pozwala na wybór polityki SCHED\_FIFO, lub SCHED\_RR. Procesy, które mają nadany stały priorytet za pomocą wywołania \emph{sched\_setscheduler()} wywłaszczają inne korzystające z metod SCHED\_OTHER, SCHED\_BATCH, oraz SCHED\_IDLE. Wywłaszczenie procesu czasu rzeczywistego odbywa się poprzez jawne wywołanie \emph{sched\_yield()}, próbę dostępu do I/O lub przez inny proces o wyższym stałym priorytecie. SCHED\_FIFO jest prostą metodą kolejkowania bez podziału czasu, zaś SCHED\_RR dodatkowo przydziela procesom czasu rzeczywistego kwanty czasu. 

Kolejnym krokiem jest utrzymanie w pamięci wszystkich stron kodu, danych i stosu związanych z daną aplikacją. W tym przypadku korzystam z funkcji \emph{mlockall()}. Aby zapewnić odpowiednią ilość miejsca na stosie, oraz ustrzec się opóźnień związanych z optymalizacją kompilatora (ang. \emph{copy-on-write}) tworzę funkcję, która alokuje zmienną automatyczną typu tablicowego. Dodatkowo, pisanie do tej zmiennej zapewnia, że cała pamięć dla niej przeznaczona będzie udostępniona przez kompilator na początku działania aplikacji.

Należy również pamiętać o tworzeniu wszelkich wątków potrzebnych do działania aplikacji na samym początku jej działania.

\subsection{Xenomai i RTAI.}

Zarówno Xenomai, jak i RTAI są rozwiązaniami opartymi o ideę współdzielenia zasobów sprzętowych. Współdzielenie odbywa się poprzez warstwę abstrakcji sprzętowej (w tym przypadku jest to nanokernel Adeos). Adeos nie jest jednak wyłącznie niskopoziomową częścią jądra, lecz pozwala na jednoczesne uruchomienie wielu jąder, które za jego pośrednictwem współdzielą zasoby sprzętowe. 

\begin{figure}[htb]
\begin{center}
\includegraphics[width=300px]{img/XenomaiRTAI}
\caption{Architektura Xenomai i RTAI}
\label{XenomaiRTAI}
\end{center}
\end{figure}

Propagacja przerwań odbywa się za pośrednictwem kolejki. Kolejka jest łańcuchem (ang. \emph{pipeline}) systemów operacyjnych, które są kolejno budzone w reakcji na otrzymane przerwanie. W przypadku Xenomai jest on umieszczony na początku kolejki i obsługuje przerwania związane z zadaniami czasu rzeczywistego. RTAI, zgodnie ze swoją polityką maksymalnej redukcji opóźnień, samodzielnie przyjmuje przerwania, a kolejki Adeos używa jedynie do dalszej propagacji nieobsłużonych przerwań (\ref{XenomaiRTAI}). 

Warto wspomnieć również o udostępnianej w środowisku Xenomai opcji \emph{skórek RTOS} (ang. \emph{real-time operating system skins}). Pozwalają one na wybór API, z którego będą korzystać uruchamiane w Xenomai aplikacje. Do wyboru są przykładowo skórki VxWorks, co ukazuje tendencje rozwoju w stronę przenośności rozwiązania.

\subsection{Porównanie Linux 2.6, Xenomai, RTAI i VxWorks.}

Z dostępnego w publikacji \cite{pub:Comparison} zestawienia systemów wynika głównie fakt, że w prostym scenariuszu, kiedy znana jest liczba (w tym przypadku wyłącznie jedna) i rodzaj pracujących aplikacji czasu rzeczywistego system Linux w wersji 2.6 spisuje się zadowalająco w roli systemu o łagodnych ograniczeniach czasowych. W przypadku jądra systemu Linux 2.6 razem z liczbą i stopniem skomplikowania uruchamianych aplikacji rośnie również ilość kodu do przeanalizowania, w celu zapewnienia całkowitego determinizmu operacji, co szybko staje się niepraktyczne.

Interesujący jest dla mnie fakt, że gdy w rolę wchodzi dodatkowo komunikacja sieciowa, systemy \emph{open-source} (RTAI i Xenomai) spisują się znacznie lepiej od VxWorks. Mniejsze opóźnienia są spowodowane wykorzystaniem modułu RTnet, który przebudowuje standardowy stos IP systemu linux pod kątem deterministycznej pracy.

%%%%%%%%%%%%%%%%%%%%%%%%%%%%

\section{Standardy 802.11 w systemach czasu rzeczywistego.}

Ze względu na swój charakter komunikacja bezprzewodowa jest nieprzewidywalna. Nie jesteśmy w stanie z góry założyć, ze sygnał nie zostanie zakłócony i informacja dotrze do celu. Pocieszający jest fakt, że na przestrzeni lat standard 802.11 podlegał wielu modyfikacjom i poprawkom (np. 802.11e, 802.11n). Część z tych aktualizacji dedykowana była możliwości zastosowania medium bezprzewodowego do komunikacji systemów czasu rzeczywistego. Koncentrują się one na potrzebie zapewnienia takiemu systemowi okien dostępu bezkolizyjnego i nadawaniu priorytetów w ruchu sieciowym. Zabiegi te pozwalają na osadzenie komunikacji systemów czasu rzeczywistego w zakłóconym paśmie transmisyjnym oraz ich koegzystencję z innymi stacjami. 

Niniejszy rozdział poświęcony jest przeglądowi dostępnych rozwinięć standardu 802.11 pod kątem użyteczności w komunikacji systemów czasu rzeczywistego.

\subsection{Problemy w 802.11 MAC - opcja DCF.}

Podstawowa wersja protokołu dostępu do medium (ang. \emph{Distributed Coordination Function}, w skrócie DCF) nie uwzględnia możliwości ustalenia priorytetów. Brak priorytetów na poziomie MAC w oczywisty sposób utrudnia realizację przewidywalnej komunikacji w systemie czasu rzeczywistego. System musiałby konkurować ze wszystkimi innymi stacjami (nawet tymi, które nie są świadome jego istnienia, a więc wprowadzają wyłącznie zakłócenia). 

Dodatkowo, stacje, które przy próbie dostępu napotkały zajęty kanał transmisji, muszą odczekać pewien losowy okres czasu (ang. \emph{Backoff}). Długość okresu oczekiwania obliczana jest jako iloczyn losowej wartości z zakresu od zera do ustalonej długości CW (ang. \emph{Contention Window}) i czasu podróży ramki w łączu (ang. \emph{Slot Time}). Oczywiste jest, że wprowadzenie losowego parametru kłóci się z ideą deterministycznej pracy.

\subsection{802.11 MAC - opcja PCF.}

Część powyżej przedstawionych problemów jest adresowana przez wprowadzenie protokołu dostępu PCF (ang. \emph{Point Coordination Function}). Opcja ta wyróżnia jedną stację (typowo jest to punkt dostępu - AP), która pełni rolę koordynatora komunikacji. Koordynator uzyskuje dostęp do łącza częściej, gdyż jego czas oczekiwania między kolejnymi ramkami jest krótszy. Po uzyskaniu dostępu do łącza koordynator wybiera stację, która może rozpocząć transmisje w oknie bezkolizyjnym. 

W tym przypadku problemem jest fakt, że nadająca stacja może wysłać ramkę arbitralnej długości. Punkt dostępowy rozsyła z okresem TBTT (ang. \emph{Target Beacon Transmission Time}) ramki typu \emph{Beacon} 
porządkujące transmisję danych. Przykładowo w 802.11e ramki \emph{Beacon} zawierają ustawienia parametrów (np. TXOP) i
informują inne stacje o zakończeniu okresu dostępu bezkolizyjnego.
Protokół dostępu PCF nie posiada ograniczeń, które mogłyby powstrzymać stację przed pogwałceniem okresu TBTT. Jest
możliwe, że stacja, która została wybrana przez AP w okresie dostępu bezkolizyjnego rozpocznie przesyłanie zbyt dużych ramek,
których czas przesyłania przekroczy czas okresu TBTT. Przekroczenie czasu TBTT powoduje, że AP opóźni propagację ramek
\emph{Beacon}. Brak możliwości deterministycznego określenia długości trwania okresu dostępu bezkolizyjnego, czy też samego
czasu transmisji pojedynczej stacji w tym okresie powoduje, że obsługa komunikacji systemów czasu rzeczywistego za pomocą
protokołu PCF jest utrudniona.

\subsection{Wsparcie dla QoS w standardzie 802.11e.}

Standard 802.11e wprowadził nowy protokół dostępu EDCA (ang. \emph{Enhanced Distributed Channel Access}). Protokół ten udostępnia 4 klasy priorytetów dla ruchu. System priorytetów zbudowany jest na bazie parametrów odziedziczonych po protokole DCF. 

Każda klasa dostępu posiada własny odstęp międzyramkowy AIFS (ang. \emph{Arbitration Interframe Space}).

Czas \emph{Backoff} nadal wyliczany jest w sposób losowy, ale w tym przypadku jest to wartość z przedziału od CWmin do CWmax (CWmin i CWmax są parametrami ustalanymi indywidualnie dla każdej klasy dostępu).

Może dojść do sytuacji, kiedy natężenie ruchu w sieci komunikacyjnej jest na tyle duże, że priorytety nie wystarczają dla zapewnienia odpowiedniej jakości połączenia dla systemu czasu rzeczywistego. W takiej sytuacji możliwe jest użycie parametru TXOP (ang. \emph{Transmission Oportunity}). 

Parametr TXOP pozwala stacji na transmisję serii ramek (ang. \emph{burst}). Po uzyskaniu dostępu do łącza
węzeł może przesyłać kolejne ramki z odstępem SIFS (ang. \emph{short interframe space}) między porcją danych, a
ramką potwierdzenia ACK. TXOP zapewnia bezkolizyjny dostęp do medium jednocześnie ograniczając maksymalny czas 
okresu dostępu bezkolizyjnego dla danej stacji. Jeśli przesyłanie ramki trwałoby dłużej niż czas TXOP to ramka
ta zostanie podzielona, aby zachować jakość usług.

Podsumowując, istnieją 4 parametry podlegające niezależnej regulacji w ramach każdej z klas dostępu:

\begin{itemize}
\item CWmin - minimalna długość losowego składnika czasu \emph{Backoff}.
\item CWmax - maksymalna długość losowego składnika czasu \emph{Backoff}.
\item AIFS - Odstęp międzyramkowy.
\item Max TXOP - Maksymalny czas dostępu bezkolizyjnego po uzyskaniu medium.
\end{itemize}

CWmin i CWmax są wykorzystywane w mechaniźmie \emph{Backoff} w celu wylosowania czasu trwania okresu wycofania stacji po
kolizji. Ostateczna długość odstępu międzyramkowego wyliczana jest poprzez sumowanie czasu SIFS z iloczynem AIFS i czasu
\emph{Slot Time} łącza. Warto zauważyć, że zmniejszanie czasów oczekiwania rywalizującej stacji umożliwia jej częstszy
dostęp do medium, lecz powoduje również wzrost kolizji na łączu. 

\subsection{Zastosowanie 802.11e w przykładowym środowisku.}

Standard 802.11e udostępnia następujące klasy dostępu:

\begin{itemize}
\item AC\_VO - klasa dostępu głosowego (najwyższy priorytet)
\item AC\_VI - klasa dostępu wideo
\item AC\_BE - klasa ruchu uprzywilejowanego 
\item AC\_BK - ruch w tle (najniższy priorytet)
\end{itemize}

W tym przypadku system czasu rzeczywistego mógłby korzystać z klasy AC\_VO (\cite{pub:802.11e}). Inne stacje świadome istnienia systemu mogą komunikować się w klasie AC\_BK. Jeśli chodzi o stacje nieświadome istnienia systemu to można dla nich przeznaczyć klasę AC\_BE.

\subsection{Rozwiązania na poziomie oprogramowania Linux.}

Mechanizm QoS (ang. \emph{Quality Of Service}) jest realizowany w jądrze 2.6 w postaci struktury komponentów, z których każdy realizuje pewien podzbiór funkcjonalności związanych z sterowaniem ruchem sieciowym (\cite{pub:QoS}). Główne składniki modułu QoS w systemie Linux to:

\begin{itemize}
\item qdisc - odpowiada za politykę kolejkowania ramek na urządzeniu.
\item class - umożliwia podział pakietów na klasy priorytetów w ramach qdisc.
\item filter - jest elementem odpowiadającym za podział na klasy lub np. upuszczanie ramek.
\end{itemize}

Wstępnie dostępne są dwa elementy qdisc - ingress i root (egress). Główna funkcjonalności skupia się w qdisc'u root, gdyż odpowiada on za kolejkowanie ramek wychodzących ( qdisc ingress umożliwia jedynie np. proste upuszczanie ramek ).

Istnieją dwa typy komponentów qdisc - korzystający z klas (ang. \emph{classfull qdisc}) i nie korzystający z klas (ang. \emph{classless qdisc}). 

Qdisc nie wykorzystujący klas pozwala na realizację prostej polityki kolejkowania typu FIFO (ang. \emph{First-in-first-out}). Standardowo, system Linux do kolejkowania swoich ramek wykorzystuje qdisc typu \emph{pfifo\_fast}, która składa się z trzech kolejek FIFO opróżnianych wedle swojego priorytetu. \emph{Classless qdisc} pozwala na realizację prostej polityki ograniczania częstotliwości ramek. TBF (ang. \emph{Token Bucket Filter}) bazuje na buforze wypełnianym małymi porcjami danych (żetonami), które są konsumowane przez wysyłane ramki.

Qdisc korzystający z klas pozwala na implementację bardzo skomplikowanych struktur drzewiastych (ang. \emph{Hierachical Token Bucket}). Każda klasa może zawierać inne klasy, przy czym w liściach drzewa następuje kształtowanie ruchu (znajdują się tam elementy qdisc). Klasy służą jedynie do odpowiedniego podziału dostępnych żetonów. Zaimplementowano również mechanizm pożyczania. Jeśli klasa dziecko wyczerpała dostępne żetony, to pożycza je od klasy rodzica.

\subsection{Stos IP RTnet.}

RTnet jest nowym stosem IP przeznaczonym dla systemów Xenomai i RTAI. Zastępuje on standardowy stos systemu Linux i wprowadza zmiany istotne z punktu widzenia komunikacji systemów czasu rzeczywistego.

Bufor pakietu \emph{sk\_buff} został zastąpiony przez strukturę \emph{rtskb}, która ma następujące własności:

\begin{itemize}
\item Stały rozmiar (zawsze maksymalny)
\item Pula buforów jest alokowana na początku działania systemu dla każdej warstwy stosu
\item Zawsze wraca do nadawcy, chyba, że odbiorca może zwrócić wskazanie na inny bufor z własnej puli
\end{itemize}

Inną ważną cechą RTnet jest fakt, że implementacja UDP/IP odbywa się poprzez statyczne przypisanie adresów (nie korzysta z protokołu ARP). Dla pakietów IP przesyłanych we fragmentach potrzebne bufory \emph{rtskb} pobierane są z puli globalnej.



\section{Pomiar czasu przełączania kanału radiowego.}
%

Ciężko uniknąć sytuacji, w której systemy wykorzystujące do komunikacji standard \emph{802.11} napotykają potrzebę zmiany częstotliwości (przełączenia kanału) pracy swoich interfejsów kart radiowych NIC (ang. \emph{Network Interface Card}). Główną przyczyną podziału pasma jest wielodostęp, a więc unikanie wzajemnego zakłócania się urządzeń. Należy wziąć pod uwagę, że medium transmisyjne w środowisku przemysłowym jest zwykle wyjątkowo zaszumione w paśmie \emph{2.4 GHz}. Dla uzmysłowienia stopnia zakłóceń wystarczy wymienić część urządzeń pracujących w paśmie \emph{ISM} (ang. \emph{Industrial, scientific and medical}) takich jak:
\begin{itemize}
\item[--] Elektroniczne nianie
\item[--] Urządzenia Bluetooth
\item[--] Kuchenki mikrofalowe
\item[--] Alarmy samochodowe
\end{itemize}
Łatwo zauważyć jak bardzo zróżnicowane urządzenia mogą doprowadzić do niespodziewanych problemów w bezprzewodowej komunikacji systemów czasu rzeczywistego.

Warto wspomnieć, że istnieje już specyfikacja standardu pracującego w paśmie \emph{5 GHz} (\cite{std:IEEE80211n}), lecz nie jest on jeszcze powszechnie wspierany. Biorąc za przykład rozwiązania \emph{open-source} można zauważyć, że standard \emph{802.11n} jest obsługiwany przez nowe sterowniki (\emph{ath9k} dla urządzeń firmy \emph{Atheros}). Problemem jest natomiast fakt, że tego typu sterowniki dostępne są jedynie w najnowszych dystrybucjach systemów operacyjnych przeznaczonych dla urządzeń wbudowanych (przykładowo \emph{OpenWrt Backfire 10.03}), które nie zawsze od początku wspierają zadowalającą gamę urządzeń. Dla przykładu nadal istnieją problemy z dostępnością tego typu sterowników dla popularnej płytki \emph{MagicBox}.

% Dodać więcej cytowań do WMN
Biorąc pod uwagę fakt zaszumienia medium transmisyjnego wnioskuję, że możliwość zmiany częstotliwości pracy interfejsu \emph{NIC} w poszukiwaniu dogodnego kanału komunikacji jest jedną z jego kluczowych i wymagających uwagi cech. W ostatnich latach powstało wiele publikacji dotyczących możliwości adaptacji struktury sieci bezprzewodowych do panującej jakości medium komunikacyjnego (\cite{pub:DCS}). Prace te koncentrują się głównie na algorytmach dynamicznej modyfikacji częstotliwości pracy interfejsów w sieciach kratowych \emph{WMN} (ang. \emph{Wireless Mesh Network}). Oczywiście u podstaw zastosowanych rozwiązań leży zjawisko przełączania kanału radiowego.

Powyższe czynniki sugerują, że całkowite wyeliminowanie potrzeby przełączania kanału (zmiany częstotliwości pracy) interfejsów radiowych nie jest aktualnie osiągalne. Co więcej, udostępnianie nowych pasm częstotliwości, w sytuacji ciągle rosnącego zapotrzebowania, jest jedynie tymczasowym rozwiązaniem.  


\subsection{Przełączanie kanału radiowego}
Opóźnienie związane ze zmianą częstotliwości pracy jest ważnym parametrem, gdyż w tym czasie stacja zaprzestaje reakcji na kierowane do niej dane. Ramki skierowane do stacji są tracone co w oczywisty sposób może wpłynąć na ograniczenia czasowe, w których działają komunikujące się systemy. 
Typowe scenariusze, w których może zajść potrzeba zmiany częstotliwości pracy interfejsu NIC to:
\begin{itemize}
\item[--] Stacja kliencka w trybie \emph{Managed} dokonuje \emph{Roamingu} między dwoma punktami dostępowymi AP (ang. \emph{Access Point}) 
\item[--] Stacja kliencka w trybie \emph{Managed} skanuje medium w poszukiwaniu punktów dostępowych AP (ang. \emph{Access Point})
\item[--] Stacja kliencka w trybie \emph{Ad-hoc} skanuje medium po podniesieniu interfejsu lub samym przełączeniu kanału 
\end{itemize}
Identyfikacja powyższych sytuacji to pierwszy krok ku specyfikacji konkretnych scenariuszy pomiarowych. 

Najczęstszą przyczyną przełączania kanału jest procedura skanowania medium komunikacyjnego. Podczas skanowania stacja wysyła ramki typu \emph{Probe Request} na każdej z dostępnych w specyfikacji (\cite{std:IEEE80211}) częstotliwości pracy i oczekuje na ramki \emph{Probe Response} od punktów dostępowych, lub stacji w trybie \emph{Ad-hoc} (w zależności od typu interfejsu NIC, czyli rodzaju docelowej sieci).

Przełączanie kanału następuje również, kiedy stacja kliencka oddala się zbyt daleko od punktu dostępowego i musi rozpocząć poszukiwanie nowego w swoim zasięgu. Jest to sytuacja zwana roamingiem i wymaga uwagi podczas rozważania systemów, w których skład wchodzą mobilne stacje, czy agenci. Obszar działania systemu może być na tyle różnorodny pod względem zakłóceń, że konieczne będzie przełączanie kanału między kolejnymi punktami dostępowymi pracującymi na różnych częstotliwościach.

\subsection{Metodyka pomiaru}
Z punktu widzenia zjawiska komunikacji w standardzie 802.11 za kluczową uznałem możliwość prowadzenia pomiarów z minimalną ingerencją w strukturę i działanie stacji. Osiągnięcie tego celu wymaga uruchomienia dodatkowej maszyny, która prowadzi nasłuch w medium komunikacyjnym. Jedną z zalet tego typu rozwiązania jest fakt, że programistyczne środowisko pomiarowe przygotowuję tylko na jednej stacji. Jest to niezwykle ważne w przypadku, gdy w danym scenariuszu pomiarowym biorą udział systemy wbudowane (np. pełniące funkcję routerów) z ograniczonymi możliwościami instalacji rozbudowanych aplikacji i bibliotek programistycznych. 
Opis stosowanych metodyk pomiarowych rozpocznę od definicji podstawowych pojęć opisujących środowisko i uczestników scenariuszy. Najważniejsze pojęcia to:
\begin{itemize}
\item[--] {\bf Stacja pomiarowa}: Komputer działający pod kontrolą interakcyjnego systemu operacyjnego, na którym uruchomiona jest aplikacja nasłuchująca ruch sieciowy (ang. \emph{sniffer}).
\item[--] {\bf Stacja kliencka}: Komputer pełniący rolę klienta w sieci o strukturze wykorzystującej punkty dostępowe (ang. \emph{Infrastructure mode}). Może być to zarówno komputer pod kontrolą systemu interakcyjnego, lub wbudowanego.
\item[--] {\bf Punkt dostępowy}: Komputer pełniący w trybie infrastruktury (ang. \emph{Infrastructure mode}) rolę stacji AP (ang. \emph{Access Point}). Może być to zarówno komputer pod kontrolą systemu interakcyjnego, lub wbudowanego.
\item[--] {\bf Rozwiązanie asocjacji}: Zdarzenie wysłania ramki rozwiązującej asocjację między stacją kliencką, a punktem dostępowym (ang. \emph{Disassociation frame}).
\item[--] {\bf Skanowanie}: Wysyłanie przez stację ramek typu \emph{Probe Request} na wszystkich dostępnych w specyfikacji (\cite{std:IEEE80211}) częstotliwościach pracy.
\item[--] {\bf Scenariusz pomiaru}: Jeden ze scenariuszy możliwych do zaistnienia podczas komunikacji stacji w standardzie 802.11, w którego czasie następuje przełączenie kanału interfejsu NIC.
\end{itemize}

\subsection{Scenariusz pomiaru: Roaming 802.11}
Roaming 802.11 to zjawisko zachodzące w sieciach, w trybie infrastruktury (ang. \emph{Infrastructure mode}). Podstawowym zadaniem procedury jest umożliwienie stacji klienckiej odłączenia się od punktu dostępowego i podjęcia próby odnalezienia i podłączenia się do stacji o mocniejszym sygnale. W warunkach rzeczywistych sytuacja taka najczęściej jest wynikiem ruchu mobilnej stacji klienckiej (np. przemieszczającego się pracownika biura, lub agenta w systemie przemysłowym), która dociera do granicy zasięgu dotychczas używanego punktu dostępowego. Aby zachować połączenie z systemem, lub usługami (np. dostęp do internetu) maszyna musi odnaleźć inną stację pracującą w trybie AP o mocniejszym sygnale. Na procedurę roamingu 802.11 składają się następujące kroki:
\begin{itemize}
\item[--] Stacja kliencka wykrywa, że poziom sygnału RF (ang. \emph{Radio Frequency})
punktu dostępowego \#1 jest poniżej progu roamingu.
\item[--] Stacja kliencka rozpoczyna nadawanie ramek rozwiązujących asocjację do punktu dostępowego \#1 do momentu potwierdzenia odebrania.
\item[--] Punkt dostępowy \#1 otrzymuje ramkę rozwiązującą asocjację \emph{Disassociation frame} i usuwa stację kliencką z tablicy asocjacji.
\item[--] Stacja kliencka rozpoczyna skanowanie medium komunikacyjnego i oczekuje na ramki typu \emph{Probe Response}.
\item[--] Punkt dostępowy \#2 wysyła do stacji klienckiej ramkę typu \emph{Probe Response}
\item[--] Stacja kliencka rozpoczyna wysyłanie do punktu dostępowego \#2 ramek typu \emph{Association request}.
\item[--] Punkt dostępowy \#2 dokonuje asocjacji stacji klienckiej i potwierdza to zdarzenie wysyłając ramkę typu \emph{Association response}.
\end{itemize}

Łatwo zauważyć, że zjawisko roamingu jest kluczowe w przypadku systemu czasu rzeczywistego zarządzającego stacjami mobilnymi na rozległym obszarze (\ref{MobileAgentSystem}). System może wykorzystywać wiele punktów dostępowych, które obsługuje poprzez sieć przewodową (ang. \emph{Ethernet}). Każda zarządzana stacja w trybie AP przystosowana jest do działania w panujących na swoim obszarze warunkach zaszumienia łącza. Roaming 802.11 byłby w tym wypadku główną przyczyną przełączania kanału radiowego interfejsu NIC w mobilnych stacjach klienckich. 

\begin{figure}[htb]
\begin{center}
\includegraphics[width=300px]{img/System_czasu_rzeczywistego}
\caption{System z mobilnym agentem}
\label{MobileAgentSystem}
\end{center}
\end{figure}

Oczywiście roaming nie implikuje ruchu żadnej z maszyn, co ułatwia przeprowadzenie pomiaru. Wystarczy doprowadzić do sytuacji, w której moc sygnału punktu dostępowego spadnie poniżej progu (ang. \emph{roaming treshold}), który powoduje decyzję o rozwiązaniu asocjacji stacji klienckiej. 

\begin{figure}[htb]
\begin{center}
\includegraphics[width=300px]{img/Roaming}
\caption{Roaming 802.11: Środowisko pomiarowe.}
\label{RoamingEnviroment}
\end{center}
\end{figure}

\subsubsection{Środowisko pomiarowe}

W skład środowiska pomiarowego (\ref{RoamingEnviroment}) wchodzą dwa punkty dostępowe, stacja kliencka oraz stacja pomiarowa. Punkty dostępowe pracują na różnych częstotliwościach. Do wyboru, zgodnie ze standardem 802.11g (\cite{std:IEEE80211}), są kanały numer 1, 5, 9, lub 13. Są to nienachodzące na siebie zakresy częstotliwości. W celu ułatwienia roamingu stacja kliencka umieszczona jest na granicy zasięgu punktów dostępowych. Stacja pomiarowa musi znajdować się w zasięgu stacji klienckiej, oraz obydwu punktów dostępowych (musi być w stanie rejestrować ruch sieciowy). 
Należy zwrócić uwagę na zapewnienie odpowiedniej jakości medium transmisyjnego. Wysoki poziom zakłóceń na kanałach wykorzystywanych w eksperymencie wprowadzi zakłamania, jeśli interesuje nas wyłącznie czas trwania samej procedury roamingu. 


\subsubsection{Mierzona wartość: Czas roamingu}
Czas roamingu 802.11 rozumiem jako czas (\ref{RoamingTime}) mierzony od momentu decyzji stacji klienckiej o zaprzestaniu normalnej wymiany danych z punktem dostępowym do momentu powiązania z nową stacją w trybie AP o mocniejszym sygnale. Zdarzeniem inicjującym pomiar jest wysłanie przez stację kliencką pierwszej ramki rozwiązującej asocjację (ang. \emph{Disassociation frame}). Pomiar zostaje zakończony w momencie wysłania przez nowy punkt dostępowy ramki potwierdzającej asocjację nowej stacji (ang. \emph{Association Response frame}).

Podstawowa procedura pomiarowa składa się z następujących kroków:
\begin{itemize}
\item[--] Stacja kliencka przeprowadza asocjację z punktem dostępowym 1.
\item[--] Stacja kliencka przemieszcza się poza zasięg punktu dostępowego 1 i wykonuje procedurę roamingu do punktu dostępowego 2.
\item[--] Stacja pomiarowa wykrywa próbę rozwiązania asocjacji i rozpoczyna pomiar czasu.
\item[--] Punkt dostępowy 2 dokonuje asocjacji stacji klienckiej.
\item[--] Stacja pomiarowa rejestruje potwierdzenie asocjacji stacji klienckiej i zatrzymuje pomiar czasu.
\end{itemize}

\begin{figure}[htb]
\begin{center}
\includegraphics[width=300px]{img/RoamingTime}
\caption{Roaming 802.11: Czas roamingu.}
\label{RoamingTime}
\end{center}
\end{figure}

\subsubsection{Wnioski}
% Wnioski na temat tego scenariusza


\section{Narzędzie pomiarowe: \emph{hop-sniffer}.}
%
 
Niniejszy rozdział poświęcony jest opisowi aplikacji powstałej na podstawie wymagań sformułowanych w opisie pomiaru (link). Analiza wymagań podyktowała stworzenie programu, który umożliwiałby pogląd ramek zarządzających komunikacją w standardzie 802.11 (ang. \emph{Management frames}) oraz analizę zależności czasowych między nimi. 

Wymagane okazało się stworzenie aplikacji nasłuchującej (ang. \emph{sniffer}) przystosowanej do obserwacji typowych scenariuszy zachodzących w komunikacji w medium bezprzewodowym. Przystosowanie to rozumiem jako możliwość konfiguracji programu pod kątem wybranego zjawiska i środowiska pomiarowego. 

\subsection{Środowisko pracy programu.}

Program hop-sniffer został przygotowany dla systemu operacyjnego Linux w wersji jądra 2.6. W wyborze systemu operacyjnego kierowałem się głównie metodą implementacji sterowników urządzeń bezprzewodowych i obsługującej je warstwy pośredniej jądra. 

System Linux był wyborem oczywistym ze względu na możliwość konfiguracji interfejsów NIC w sposób umożliwiający przetwarzanie ramek typu MGMT (ang. \emph{management}) standardu 802.11 za pomocą aplikacji w przestrzeni użytkownika.

Kolejną zaletą wybranego systemu jest możliwość konfiguracji najbardziej odpowiedniej dystrybucji i kompilacji powstałego rozwiązania jedynie z użyciem opcji dedykowanych dla aplikacji pomiarowej. W tym wypadku najbardziej pożądane jest minimalistyczne środowisko, które w możliwie najmniejszym stopniu wpływało będzie na prezentowane przez program wyniki pomiarów. Jako środowisko zalecane wybrałem system Arch Linux (link).

Biorąc pod uwagę program komunikujący się z kartą radiową w systemie Linux należy zwrócić szczególną uwagę na kwestię sterowników. Od sterowników urządzeń bezprzewodowych zależy jakie polecenia i tryby pracy interfejsów będą dostępne do konfiguracji w przestrzeni użytkownika. Ze względu na aktualne dążenie programistów jądra do unifikacji interfejsu obsługi urządzeń standardu 802.11 powstała warstwa pośrednia \emph{mac80211}. Postanowiłem oprzeć aplikację hop-sniffer o sterowniki działające w tej warstwie ze względu na wspólny, oparty na gniazdach interfejs komunikacyjny \emph{nl80211}. Kluczowym wymaganiem stawianym sterownikowi jest implementacja polecenia umożliwiającego utworzenie wirtualnego interfejsu karty radiowej pracującego w trybie \emph{monitor}.

Wprowadzenie karty radiowej w tryb \emph{promiscuous} powoduje jedynie wyłączenie filtracji adresów MAC. Program hop-sniffer musi mieć możliwość odbierania ramek standardu 802.11 bez potrzeby asocjacji z SSID (ang. \emph{Service Set Identifier}) żadnej sieci. Wyłączenie filtracji SSID możliwe jest jedynie w trybie \emph{monitor}.

W swojej pracy skoncentrowałem się na współpracy ze sterownikiem \emph{ath9k}. Jest to całkowicie otwarty sterownik do urządzeń standardu 802.11bgn firmy \emph{Atheros}. Za wykorzystaniem sterownika przemawia dostępność wspieranych przez niego urządzeń, implementacja szerokiej gamy poleceń interfejsu \emph{nl80211} oraz możliwość pracy w trybie \emph{monitor}.

\subsection{Biblioteki programistyczne.}

Implementacja aplikacji pomiarowej wymagała zastosowania API (ang. \emph{Application interface}) umożliwiającego pochwycenie ramek zarządzających komunikacją 802.11 w przestrzeni użytkownika. Podczas procesu tworzenia programu hop-sniffer rozpatrzyłem zastosowanie dwóch bibliotek: \emph{libnl} i \emph{libpcap}.

\subsubsection{Nasłuchiwanie za pomocą interfejsu \emph{nl80211}.}

\emph{Libnl} jest to API (ang. \emph{Application interface}) służące do komunikacji między przestrzenią użytkownika i warstwą \emph{mac80211} jądra systemu operacyjnego. Interfejs \emph{nl80211} tej warstwy oparty jest o system gniazd \emph{Generic Netlink} (w odróżnieniu od stosowanych dawniej wywołań systemowych \emph{IOCTL}).  

Warstwa pośrednia definiuje rodzinę gniazd (ang. \emph{Generic netlink family}) oraz rejestruje w jej obrębie zestaw poleceń w postaci akceptowanych rodzajów wiadomości. Sterowniki urządzeń 802.11 implementują powyższy interfejs poprzez inicjalizację odpowiadających poleceniom wskaźników na funkcje własnymi operacjami. Każda wiadomość akceptowana przez daną rodzinę posiada własną nazwę oraz wskaźnik na strukturę określającą ilość i typy atrybutów (ang. \emph{Generic netlink attribute policy}), która pełni funkcję kontroli poprawności. Struktura ta zwana \emph{nla\_policy} stanowi wytyczne co do sposobu konstrukcji skierowanego do jądra polecenia oraz ekstrakcji danych z odebranej wiadomości.

W wyniku analizy dokumentacji uznałem, że możliwa będzie implementacja programu nasłuchującego z wykorzystaniem następujących mechanizmów udostępnianych przez interfejs \emph{nl80211}:

\begin{itemize}
\item[--] Grupowych adresów (ang. \emph{Multicast groups}) odbiorców wiadomości.
\item[--] Komendy \emph{NL80211\_CMD\_REGISTER\_FRAME}.
\item[--] Własnych funkcji obsługi zdarzeń (ang. \emph{Custom callback}).
\item[--] Komendy \emph{NL80211\_CMD\_FRAME}.
\end{itemize}

Adresy grupowe są wykorzystywane przez jądro do rozgłaszania zdarzeń warstwy \emph{mac80211} do zainteresowanych procesów (posiadających gniazdo z członkostwem w danej grupie rozgłaszania). W celu otrzymywania wszystkich zdarzeń należy zarejestrować gniazdo (\ref{code:MulticastExample}) we wszystkich czterech grupach: \emph{Configuration, Scan, Regulatory i MLME}. 

\lstset{caption={Przykład rejestracji gniazda w grupie \emph{Configuration}.}, label={code:MulticastExample}}
\begin{lstlisting}[frame=tb]
/* Get configuration multicast group ID */
multicast_id = nl_get_multicast_id(cd->nl_sock, 
        "nl80211", "config");
if (multicast_id < 0)
        return multicast_id;
                                        
/* Add membership to configuration multicast group */
ret = nl_socket_add_membership(cd->nl_sock, multicast_id);
if (ret)
        return ret;
\end{lstlisting}

Komenda \emph{NL80211\_CMD\_REGISTER\_FRAME} pozwala na rejestrację wybranych typów ramek do przetwarzania w przestrzeni użytkownika. Wymaga podania numeru interfejsu radiowego, typu ramki oraz wzorca zawierającego pierwsze bajty ramki, które powinny być dopasowane. Należy wziąć pod uwagę fakt, że w tym wypadku nasza aplikacja musi obsłużyć dany typ ramek, gdyż nie zostaną one odpowiednio przetworzone w jądrze. Zamknięcie gniazda komunikacyjnego za pomocą, którego dokonano rejestracji powoduje jej zakończenie. 

W mojej aplikacji zgłaszanie ramek do obsługi przez program nasłuchujący jest częścią inicjalizacji. Należy zarejestrować wszelkie ramki niezbędne do obserwacji wybranego zjawiska. Proces budowania wiadomości (\ref{code:RegisterFrame}) zaczyna się od stworzenia nagłówka opatrzonego odpowiednim adresem odbiorcy (identyfikatorem rodziny) oraz nazwą polecenia do wykonania. Następnie dodaję atrybuty wybranej komendy podając jej identyfikator (potrzebny w celu sprawdzenia poprawności) oraz wartość. Numer interfejsu tłumaczony jest z nazwy (np. \emph{wlan0}) na indeks (typ całkowity). Rodzaj ramki to wartość typu \emph{u16}, a więc liczba 16-bitowa, którą w języku C możemy wprowadzić, przykładowo, jako \emph{0x0040} (ramka typu \emph{Probe Request}). 

\lstset{caption={Przykład rejestracji ramki do obsługi w przestrzeni użytkownika.}, label={code:RegisterFrame}}
\begin{lstlisting}[frame=tb]
/* Build netlink message header */
genlmsg_put(msg, 0, 0, genl_family_get_id(cd->nl80211), 
        0, 0, NL80211_CMD_REGISTER_FRAME, 0);
/* Device interface index to use */
devid = if_nametoindex(if_name);
NLA_PUT_U32(msg, NL80211_ATTR_IFINDEX, devid);
/* Register frame type/subtype */
NLA_PUT_U16(msg, NL80211_ATTR_FRAME_TYPE, fr_type);
/* Frame match for MGMT frames is NULL */
NLA_PUT(msg, NL80211_ATTR_FRAME_MATCH, 0, NULL);       
/* Send message */
error = nl_send_auto_complete(cd->nl_sock, msg);
\end{lstlisting}

Po przyłączeniu gniazda do odpowiednich grup rozgłaszania i wybraniu niezbędnych typów ramek do przetwarzania przez program pozostaje rozpocząć nasłuchiwanie zdarzeń interfejsu \emph{nl80211} (\ref{code:ListenEvents}). W mojej aplikacji wybór badanego zjawiska (sposobu reakcji na zdarzenia) zależy od rodzaju funkcji do której wskaźnik jest przekazywany podczas rozpoczęcia nasłuchu. Funkcja ta przekazywana jest jako argument procedury typu \emph{callback} używanej do przetwarzania odebranych wiadomości (zdarzeń). 

\lstset{caption={Fragment kodu procedury rozpoczynającej obsługę zdarzeń.}, label={code:ListenEvents}}
\begin{lstlisting}[frame=tb]
/* Choose scenario type. */
args.handle_frame = fptr_handle_frame;
/* ... */
/* set custom event handler and pass arguments to it. */
nl_cb_set(cb, NL_CB_VALID, NL_CB_CUSTOM, custom_event_handler, &args);
/* ... */
/* Listen events. */
while (!command)
{
        nl_recvmsgs(cd->nl_sock, cb);
}
\end{lstlisting}

Komenda \emph{NL80211\_CMD\_FRAME} służy do nadawania i odbierania wybranych typów ramek z poziomu aplikacji użytkownika. W przypadku programu nasłuchującego interesuje mnie funkcjonowanie tej wiadomości jako zdarzenia propagowanego przez jądro w sytuacji otrzymania nieobsłużonej ramki. Powoduje to uruchomienie własnej procedury \emph{callback}, która na wejściu otrzymuje wskaźnik do wiadomości opisującej zdarzenie oraz przekazaną przez program główny strukturę argumentów zawierającą wskaźnik do funkcji, która powinna być użyta do obserwacji danego zjawiska. 

Zdarzenie typu \emph{NL80211\_CMD\_FRAME} posiada interesujący mnie atrybut reprezentujący całą otrzymaną ramkę. Atrybut \emph{NL80211\_ATTR\_FRAME} przekazywany jest do funkcji obsługującej dane zjawisko (\emph{handle\_frame}), gdzie jest on rozpakowywany do postaci tablicy bajtów. Tablica ta służy do odczytania adresów MAC o rozmiarze sześciu bajtów (adres źródłowy z dziesiątego bajtu i adres docelowy z bajtu czwartego) oraz typu ramki z pola kontrolnego za pomocą maski bitowej \emph{0xfc}. W przypadku funkcji \emph{handle\_frame} obserwowanym zjawiskiem są ramki typu \emph{Disassociation} i \emph{Association Response}, które oznaczają odpowiednio rozpoczęcie i zakończenie procesu roamingu stacji klienckiej.

\lstset{caption={Funkcja \emph{handle\_frame}.}, label={code:HandleFrame}}
\begin{lstlisting}[frame=tb]
void handle_frame(struct nlattr *nl_frame)
{
	uint8_t *frame;
	size_t len;
	int i;
	char macbuf[6*3];
	uint16_t tmp;
        
        /* ... */

        /* Extract frame byte array from netlink attribute */
	frame = nla_data(nl_frame);
        
        /* ... */

	switch (frame[0] & 0xfc) 
        {
	        case 0x10: /* assoc resp */
	        case 0x30: /* reassoc resp */
                case 0x00: /* assoc req */
                        printf("[assoc req]\n");
                        break;
                case 0x20: /* reassoc req */
                case 0x40: /* probe req */
                case 0x50: /* probe resp */
                case 0xb0: /* auth */
                case 0xa0: /* disassoc */
                        printf("[disassoc]\n");
                        break;
                case 0xc0: /* deauth */
                        break;
        }

}
\end{lstlisting}

Stworzona przeze mnie aplikacja oparta na powyższej opisanej metodyce spełniała założenia powstałe w fazie analizy wymagań dla testowanych ramek standardu 802.11 typu \emph{Probe Request}. Niestety rejestracja ramek dla interfejsu typu \emph{monitor} okazała się niemożliwa, a typy ramek możliwe do odbierania na poszczególnych interfejsach (ang. \emph{Supported RX frame types}) są całkowicie zależne od implementacji sterownika i mocno ograniczone ze względu na jego typ. Aktualnie typy ramek 802.11 możliwe do wysyłania i dobierania na danym interfejsie są dostępne i ogłaszane w atrybutach wirtualnego urządzenia reprezentującego kartę radiową (ang. \emph{Wiphy}). Urządzenie to jest zaimplementowane w warstwie pośredniej \emph{mac80211}, a struktury je opisujące wypełniane są przez odpowiadający mu sterownik.

Inspekcja możliwych do obsługi w przestrzeni użytkownika ramek możliwa jest dzięki analizie odpowiedzi interfejsu \emph{nl80211} na komendę \emph{NL80211\_CMD\_GET\_WIPHY} z dodatkową flagą nagłówka \emph{netlink} o nazwie \emph{NLM\_F\_DUMP}, która powoduje przekazanie do wysyłającej aplikacji wiadomości ze wszystkimi parametrami wybranych urządzeń \emph{Wiphy}.

\lstset{caption={Część atrybutów \emph{Wiphy} o identyfikatorze \emph{phy0} (program \emph{iw-3.2}).}, label={code:WiphyAttributes}}
\begin{lstlisting}[frame=tb]
marcin@marcin-PC:~/iw-3.2$ ./iw phy0 info
Wiphy phy0
        ...
	Supported interface modes:
		 * IBSS
		 * managed
		 * AP
		 * AP/VLAN
		 * WDS
		 * monitor
		 * mesh point
		 * P2P-client
		 * P2P-GO
	software interface modes (can always be added):
		 * AP/VLAN
		 * monitor
	interface combinations are not supported
        ...
	Supported RX frame types:
		 * IBSS: 0x00d0
		 * managed: 0x0040 0x00d0
		 * AP: 0x0000 0x0020 0x0040 0x00a0 0x00b0 
                   0x00c0 0x00d0
		 * AP/VLAN: 0x0000 0x0020 0x0040 0x00a0 
                   0x00b0 0x00c0 0x00d0
		 * mesh point: 0x00b0 0x00c0 0x00d0
		 * P2P-client: 0x0040 0x00d0
		 * P2P-GO: 0x0000 0x0020 0x0040 0x00a0 
                   0x00b0 0x00c0 0x00d0
	Device supports RSN-IBSS.
\end{lstlisting}

Analiza możliwych do odebrania ramek wskazuje, że nie jest możliwa analiza zjawiska roamingu. Interfejsy nie pozwalają na rejestrację w jądrze ramek typu \emph{Association Response} (identyfikator \emph{0x1}), a odbieranie ramek typu \emph{Disassociation} (identyfikator 0xA) wymaga wprowadzenia interfejsu w tryb \emph{Master} (uruchomienia programu \emph{hostapd}, a więc utworzenia na komputerze punktu dostępowego). 

Oczywiście, jeśli nasłuchiwanie nie będzie prowadzone w trybie \emph{monitor} to program i tak nie otrzyma ramek z sieci, której nie jest członkiem (ze względu na filtrację SSID). 

Powyższe problemy powodują, że mimo uniwersalności i licznych zalet związanych z prostym sposobem ekstrakcji danych z ramek standardu 802.11 biblioteka \emph{Libnl} nie nadaje się do zastosowania w aplikacji opisanej wymaganiami sformułowanymi w fazie opisu procedury pomiarowej (link). 

\subsection{Nasłuchiwanie za pomocą biblioteki typu \emph{pcap}.}

 


\section{Wnioski z pomiaru roamingu \emph{802.11}.}
%

Niniejszy rozdział poświęcony jest opisowi faktycznej realizacji planowanego eksperymentu pomiarowego. Analizowane scenariusze są próbą symulacji różnych warunków, w których może dochodzić do roamingu stacji klienckiej. Manipuluję takimi parametrami jak:
\begin{itemize}
\item[--] Stosowana metoda uwierzytelniania stacji klienckiej.
\item[--] Wielkość różnicy częstotliwości między punktami dostępowymi.
\item[--] Moc sygnału \emph{TX} punktów dostępowych. 
\item[--] Obecność innych stacji pracujących na tym samym kanale.
\end{itemize}

W celu najbardziej wyraźnego zobrazowania wpływu metody uwierzytelniania wybrałem dwie skrajne techniki:
\begin{itemize}
\item[--] Otwarty system bez uwierzytelniania.
\item[--] Uwierzytelnianie \emph{WPA2-PSK} z szyfrowaniem \emph{CCMP}.
\end{itemize}

W opcji pierwszej obydwa punkty dostępowe zezwalają każdej stacji na natychmiastową asocjację. Uwierzytelnianie \emph{WPA2} natomiast wykorzystuje bardziej czasochłonne operacje wymagające wymiany danych między \emph{AP} i stacją kliencką (np. \emph{Four Way Handshake}). W eksperymencie z użyciem uwierzytelniania dodatkowo stosuję również szyfrowanie blokowe \emph{CCMP} (ang. \emph{Counter Mode with Cipher Block Chaining Message Authentication Code Protocol}). Hasło w postaci \emph{pass-phrase} jest wspólne dla obydwu punktów dostępowych i stacji klienckiej.

Specyfikacja eksperymentu zakłada, że wszystkie uczestniczące w nim stacje pracują na nienachodzących na siebie kanałach 1, 5, 9 lub 13. Dodatkowo przeprowadzam pomiar dla stacji pracujących na sąsiednich kanałach 5 i 6 w celu wprowadzenia dodatkowych zakłóceń związanych z wykorzystaniem nachodzących na siebie kanałów.

Podczas roamingu ważnym parametrem jest moc sygnału punktów dostępowych. W zależności od usytuaowania obszarów wpływu, przełączanie między nimi może zachodzić w zróżnicowanych warunkach mocy sygnału z punktu widzenia stacji klienckiej. Sytuacja taka zachodzi, gdyż roaming 802.11 jest zjawiskiem występującym na skraju zasięgu punktów dostępowych. Ze względu na usytuowanie uczestników pomiaru najbardziej skutecznym i zastosowanym przeze mnie sposobem ograniczenia siły sygnału jest wykręcenie anten obsługujących \emph{pigtail MAIN} przy jednoczesnej drastycznej redukcji parametru \emph{power} przynależnych im interfejsów radiowych.

Ostatecznie badam wpływ obecności stacji nie przewidzianych w specyfikacji scenariusza. W tym celu wykorzystuję pracujące w medium transmisyjnym punkty dostępowe nie będące uczestnikami pomiaru. Ustawiam ich kanały pracy na częstotliwości wykorzystywane w eksperymencie. 

\section{Stan medium transmisyjnego.}

Eksperyment pomiarowy przeprowadzam w warunkach domowych miejskich. W czasie pomiarów widoczny był sygnał z pięciu punktów dostępowych o zróżnicowanej mocy i częstotliwości pracy:

\begin{enumerate}
\item Częstotliwość: 2412; sygnał: -89.00 dBm; ostatnio wykryty: 704 ms temu; kanał 1
\item Częstotliwość: 2437; sygnał: -61.00 dBm; ostatnio wykryty: 384 ms temu; kanał 6
\item Częstotliwość: 2437; sygnał: -86.00 dBm; ostatnio wykryty: 316 ms temu; kanał 6
\item Częstotliwość: 2452; sygnał: -90.00 dBm; ostatnio wykryty: 248 ms temu; kanał 9
\item Częstotliwość: 2462; sygnał: -64.00 dBm; ostatnio wykryty: 72 ms temu; kanał 11
\end{enumerate}

Jest to typowa sytuacja, z którą można się spotkać w bloku mieszkalnym. Część mieszkań posiada działające \emph{AP} pracujące w różnej odległości od środowiska pomiarowego. W celu realizacji scenariusza przewidującego minimalne zakłócenia nie wykorzystuję kanału szóstego o największej zajętości.

\section{Testowane modele kart radiowych i systemów.}

Zorganizowane środowisko pomiarowe \ref{TestEnviroment} jest zgodne ze specyfikacją \ref{sec:MeasurementEnviroment} i przedstawia się następująco:
\begin{itemize}
\item[--] {\bf Punkt dostępowy 1:} Stacja \emph{PC} pod kontrolą systemu \emph{Linux} w wersji jądra 2.6.
\item[--] {\bf Punkt dostępowy 2:} Router \emph{802.11g} pod kontrolą wbudowanego systemu operacyjnego.
\item[--] {\bf Stacja kliencka:} Komputer przenośny pod kontrolą systemu \emph{Windows XP}.
\item[--] {\bf Stacja pomiarowa:} Komputer przenośny pod kontrolą systemu \emph{Linux} w wersji jądra 2.6.
\end{itemize}

\begin{figure}[htb]
\begin{center}
\includegraphics[width=250px]{img/TestEnviroment}
\caption{Środowisko pomiarowe.}
\label{TestEnviroment}
\end{center}
\end{figure}

Stacja \emph{PC} wyposażona jest w kartę radiową \emph{PCI-Express 802.11bgn} obsługiwaną przez sterownik \emph{ath9k} z \emph{chipsetem} \emph{AR9285} firmy \emph{Atheros} pozwalającą na utworzenie interfejsu w trybie \emph{AP} na bazie warstwy \emph{mac80211}. Funkcjonalność punktu dostępowego realizowana jest w przestrzeni użytkownika przez demona \emph{hostapd}, który steruje asocjacją i uwierzytelnianiem stacji klienckich. 

Rolę punktu dostępowego docelowego (numer 2) pełni router \emph{802.11bg TP-Link TL-WR543G}. Działa on pod kontrolą systemu wbudowanego i umożliwia konfigurację parametrów \emph{AP} poprzez interfejs sieciowy. 

Stacja kliencka wykorzystuje zarządce połączeń bezprzewodowych \emph{WZC} (ang. \emph{Wireless Zero Configuration}) dostępnego na systemach \emph{Windows XP}. Karta radiowa oparta jest o \emph{chipset Realtek RT8187B}. Zarządca pozwala ustawienie preferowanych punktów dostępowych i automatyczny roaming w razie osłabienia jakości łącza. 

Stacja pomiarowa jest zgodna z projektem środowiska wykonawczego programu \emph{hop-sniffer} \ref{sec:ProgramEnviroment}. Jest uruchomiona na komputerze przenośnym \emph{ASUS eeePC} z kartą radiową \emph{802.11bgn} obsługiwaną przez sterownik \emph{ath9k} posiadającą \emph{chipset} \emph{Atheros Communications Inc. AR9285}.

Stacje umieszczone są we wspólnym pomieszczeniu.

\section{Metody uśredniania wyników.}

Ze względu na niedeterministyczne opóźnienia niezbędne jest wstępne zapoznanie się z charakterem wyników i określenie najbardziej odpowiedniej metodyki ich uśredniania. 

Wstępna analiza puli wyników wskazuje na okresowe pojawianie się wartości odstających. Opierając się na znajomości charakterystyki komunikacji w standardzie 802.11 uznaję te wartości za chwilowe zakłócenia łącza, które powodują utratę części pakietów używanych do zawiązania asocjacji między punktem dostępowym i stacją kliencką. Oczywiście nie są to wartości, które można całkowicie zignorować, gdyż taka decyzja doprowadziłaby do utracenia faktu wrażliwości łącza bezprzewodowego na zakłócenia. W tej sytuacji postanowiłem posiłkować się rozwiązaniami statystycznymi stosując zarówno \emph{średnią arytmetyczną} jak i \emph{medianę} wyników. 

Średnia arytmetyczna przedstawia mniej \emph{optymistyczną} wizję opóźnień uwzględniając duży wpływ szczególnie wysokich wartości odstających. \emph{Mediana} pomaga mi w obrazowaniu najbardziej prawdopodobnych wartości i umożliwia bardziej ogólne wnioskowanie na ich podstawie.

\section{Wnioski na temat wyników pomiaru.}

Wykonałem następujące scenariusze pomiarowe:
\begin{itemize}
\item[--] {\bf SC1}: Uwierzytelnianie WPA2-PSK, przełączanie z kanału 9 na 13, moc sygnału interfejsów \emph{AP} 20 dBm.
\item[--] {\bf SC2}: System otwarty, przełączanie z kanału 9 na 13, moc sygnału interfejsów \emph{AP} 20 dBm.
\item[--] {\bf SC3}: System otwarty, przełączanie z kanału 5 na 6, moc sygnału interfejsów \emph{AP} 20 dBm.
\item[--] {\bf SC4}: System otwarty, przełączanie z kanału 5 na 6, moc pracy interfejsów \emph{AP} obniżona (na routerze do minimum, a w punkcie dostępowym zorganizowanym na komputerze \emph{PC} do wartości ułamkowej), wykręcone anteny.
\item[--] {\bf SC5}: Uwierzytelnianie WPA2-PSK, przełączanie z kanału 5 na 6, moc pracy interfejsów \emph{AP} obniżona (na routerze do minimum, a w punkcie dostępowym zorganizowanym na komputerze \emph{PC} do wartości ułamkowej), wykręcone anteny.
\item[--] {\bf SC6}: Uwierzytelnianie WPA2-PSK, brak przełączania kanału (obydwa punkty pracują na kanale 9), moc sygnału interfejsów \emph{AP} 20 dBm.
\end{itemize}

Każdy scenariusz składa się z 10 próbek, wyniki przedstawiłem w tabeli \ref{tab:Results}.

W pierwszym przypadku pomiarowym SC1 z użyciem uwierzytelniania \emph{WPA2-PSK} mediana i średnia arytmetyczna wyników są do siebie zbliżone. Oznacza to, że wartości mierzone były zbierane przy minimalnym stopniu zakłóceń w łączu. Wynik ten nie jest zaskakujący, gdyż punkty dostępowe działają z mocą nadawania 20 \emph{dBm} na częstotliwościach o niskim zatłoczeniu.

Przypadek drugi SC2 został przeprowadzony w warunkach maksymalnie zbliżonych do swojego poprzednika (SC2) z tą różnicą, że nie używane są metody uwierzytelniania stacji klienckiej. Pocieszający jest tutaj fakt zbliżenia mediany i średniej arytmetycznej (mało wartości odstających), gdyż umożliwia on porównanie uzyskanego wyniku ze scenariuszem wykorzystującym uwierzytelnianie \emph{WPA-PSK}. Widoczny jest czas jaki stacje poświęcają na negocjację (\emph{Four Way Handshake}) kluczy zabezpieczeń. Duże prawdopodobieństwo niskiego poziomu zakłamań pozwala przypuszczać, że różnica średniej arytmetycznej opóźnień przypadku pomiarowego SC2 i SC1 jest czasem jaki punkt dostępowy poświęca na uwierzytelnianie asocjującego klienta. W tym przypadku wynik \emph{20.48} milisekund to w przybliżeniu jedna trzecia czasu poświęcanego na procedurę roamingu (\emph{59.88 ms}) co ukazuje ogromny narzut algorytmów zabezpieczeń sieci bezprzewodowych na parametry czasowe komunikacji.

W scenariuszu SC3 nie jest wykorzystywana żadna metoda uwierzytelniania. Koncentruje się on na wprowadzeniu zakłóceń do medium transmisyjnego w postaci innych stacji. Po pierwsze, na kanale docelowym 6 pracują już dwa nie związane z pomiarem punkty dostępowe. Dodatkowe zakłócenia wprowadza również początkowa stacja \emph{AP} pracująca na nachodzącym kanale 5. Pojawia się znaczna różnica między medianą i średnią arytmetyczną pomiarów co potwierdza przypuszczenia co do faktu związania wartości odstających z poziomem zaszumienia łącza. Zwiększenie opóźnień wynika z faktu gubienia i retransmisji części ramek składających się na przebieg roamingu stacji klienckiej. Przypadek ten służy głównie jako porównanie z poprzednim pomiarem SC2. Skutecznie udowadnia on, że wzrost zaszumienia łącza może doprowadzić do zwiększenia opóźnień roamingu.

Przypadek SC4 został wybrany w celu zobrazowania wpływu obniżenia mocy sygnału punktów dostępowych. Ze względu na fakt, że uczestnicy pomiaru znajdowali się we wspólnym pomieszczeniu o niewielkich rozmiarach, programistyczna manipulacja mocą nadawania interfejsów radiowych miała znikomy wpływ na poziom sygnału odbierany przez stację kliencką. Dopiero próba obniżenia sygnału do wartości ułamkowej \emph{dBm} wprowadziła widoczne osłabienie. Aby symulować znaczne pogorszenie tego parametru zdecydowałem się na usunięcie anten z używanych urządzeń. Krok ten spowodował natychmiastowy spadek sygnału do odpowiednio niskich wartości zbliżonych do sytuacji zauważalnego oddalenia uczestników scenariusza. Zaobserwowałem wyjątkowo wysoki poziom gubienia ramek, który miejscami utrudniał zebranie wystarczającej liczby pomiarów. Wartości charakteryzują się bardzo dużym rozrzutem opóźnień (blisko czterokrotna różnica mediany i średniej arytmetycznej). Przewiduję, że przy zakłóceniach tego stopnia subtelny wpływ metody uwierzytelniania, czy nawet zajętości kanału staje się niezauważalny i niemożliwy do trafnej analizy. Wniosek ten sugerowany jest wynikiem kolejnego przypadku pomiarowego SC5, który dla tych samych warunków wprowadza uwierzytelnianie \emph{WPA2-PSK}. Wbrew przypuszczeniom opóźnienie jest mniejsze. Według mnie ilość pomiarów wykonanych przez program w tym scenariuszu nie uchwyciła stopnia zmienności czasu roamingu stacji klienckiej co zaowocowało niemożnością ustalenia wartości odstających i średnich. 

Ostatecznie wykonałem procedurę pomiarową SC6, w której stacja kliencka przełącza się między dwoma punktami dostępowymi pracującymi na tej samej częstotliwości. Krok ten miał w zamierzeniu umożliwić oszacowania interesującego mnie czasu przełączania kanału radiowego. Należy wziąć pod uwagę, że stacje w trybie \emph{AP} pracujące w tym samym kanale zakłócają się wzajemnie. Możliwe jest jednak wzięcie poprawki na ten typ zakłamania dzięki analizie przypadków SC2 i SC3, które obrazują możliwy, spodziewany wzrost opóźnień związany ze zbliżonymi częstotliwościami pracy punktów dostępowych. Biorąc pod uwagę medianę pomiaru SC6 \emph{64.27 ms} i SC1 \emph{82.05 ms} wnioskuję, że czynnikami składającymi się na różnicę tych wartości (\emph{17.78} ms) są zakłócenia częstotliwościowe i szukana wartość czasu zmiany kanału pracy. Niestety różnica median pomiarów SC2 \emph{55.27 ms} i SC3 \emph{83.83 ms} wynosi aż \emph{28.56 ms} co wskazuje na fakt, iż wahania opóźnień związane z zakłóceniami wynikającymi ze zbliżonych częstotliwości są zbyt duże, aby możliwa była ekstrakcja czasu przełączania kanału. 

\begin{sidewaystable}
\caption{Wyniki scenariuszy pomiarowych.}
\label{tab:Results}
\begin{tabular}{ | l || l | l | l | l | l | l | }
\hline                       
Pomiar & SC1 & SC2 & SC3 & SC4 & SC5 & SC6 \\ 
Uwierzytelnianie & WPA2-PSK & Open sys. & Open sys. & Open sys. & WPA2-PSK & WPA2-PSK \\
Kanał początkowy & 9 & 9 & 5 & 5 & 5 & 9 \\
Kanał końcowy & 13 & 13 & 6  & 6 & 6 & 9 \\
Inne stacje & Nie & Nie & Tak & Tak & Tak & Nie \\
Obniżona moc & Nie & Nie & Nie & Tak & Tak & Nie \\ \hline \hline
Średnia arytmetyczna [ms] & 80.36 & 59.88 & 75.87 & 428.91 & 86.09 & 81.26 \\ \hline \hline
Mediana [ms] & 82.05 & 55.27 & 83.83 & 101.35 & 91.38 & 64.27 \\
\hline  
\end{tabular}

\end{sidewaystable}









\section{Kierunki rozwoju.}
% Podsumowanie

Na podstawie zaprezentowanych wyników pomiarów dokonałem analizy wpływu stanu medium transmisyjnego oraz ustawień interfejsów poszczególnych uczestników pomiaru na zjawisko roamingu stacji klienckiej w standardzie 802.11. Moim dążeniem była identyfikacja głównych czynników będących przyczyną zwiększenia opóźnień. Dobrane parametry różnicujące scenariusze pomiarowe miały za zadanie symulację sytuacji, które może napotkać stacja przemieszczająca się między strefami zasięgu punktów dostępowych.

Powstała w wyniku pracy aplikacja \emph{hop-sniffer} posiada szerokie możliwości rozwoju. Skuteczna przeprowadzenie obserwacji opisywanego zjawiska dowodzi użyteczności zastosowanego podejścia. Ostatecznie implementacji uległa część dotycząca roamingu, ale program oferuje bazę do wdrożenia innych procedur pomiarowych, pod warunkiem, że ich elementarnymi krokami są zdarzenia przesłania ramek standardu 802.11. Uniwersalność rozwiązania może zostać zwiększona dzięki zastosowaniu interfejsu użytkownika do komponowania niezdefiniowanych scenariuszy. Cel ten może zostać łatwo osiągnięty przez wprowadzenie dodatkowej abstrakcji procedury pomiarowej, która sprowadza się do określenia różnicy stempli czasowych wybranych zdarzeń (określanych przez typy ramek i ich zawartość).Aplikacja ukazuje również sposób podejścia do potrzeby przełączania kanału radiowego w trakcie realizacji scenariusza bez groźby zgubienia ramek potrzebnych do jego kontynuacji. 

Ze względu na dużą ilość czynników wpływających na zakłócenia łącza i czasy komunikacji bardziej przewidywalne byłoby izolowane środowisko pomiarowe. Przykładowo, wszystkie cztery stacje biorące udział w pomiarze mogą znajdować się w ekranowanym opakowaniu, gdzie wymuszenie roamingu odbywałoby się poprzez zdecydowane osłabienie mocy sygnału jednego z punktów dostępowych. Oczywiście pojęcie \emph{stacja} nie musi oznaczać pojedynczego komputera, lecz interfejs karty radiowej zamontowanej na wspólnej maszynie. W takich warunkach wpływ poszczególnych czynników jest łatwiejszy do określenia. 






\nocite{*}
\bibliography{bibliografia}
\end{document}
